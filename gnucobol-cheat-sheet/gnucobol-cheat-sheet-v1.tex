\documentclass[10pt,landscape]{article}
\usepackage{../common/cheat-style}
\usepackage{url}
\usepackage{hyperref}

\begin{document}

\setlength{\premulticols}{1pt}
\setlength{\postmulticols}{1pt}
\setlength{\multicolsep}{1pt}
\setlength{\columnsep}{2em}

\def\csRevision{1}
\def\csSubtitle{Cobol 85, 2002, 2014}
\makeheader{GnuCOBOL Compiler}

\vspace{-2mm}
\begin{multicols}{3}

  \vbox{
   ~\\ %space under BY-SA logo
\begin{itemize}
\item  GnuCOBOL website: \url{https://gnucobol.sourceforge.io/}
\item  GnuCOBOL Manual: \url{https://gnucobol.sourceforge.io/doc/gnucobol.html}
\item  GnuCOBOL Language Quick Reference: \url{https://gnucobol.sourceforge.io/HTML/gnucobqr.html}
\item  Sample Programs: \url{https://gnucobol.sourceforge.io/HTML/gnucobsp.html}
\item  FAQ: \url{https://gnucobol.sourceforge.io/faq/index.html}
\end{itemize}


\subsection{Installation}

{\bf Source repositories:}\vspace{-3mm}
\begin{verbatim}
REPO=https://svn.code.sf.net/p/gnucobol/code
svn checkout $REPO/branches/gnucobol-3.x
\end{verbatim}\vspace{-3mm}
{\bf or:}\vspace{-3mm}
\begin{verbatim}
REPO=https://svn.code.sf.net/p/gnucobol/code
svn checkout $REPO/code/trunk gnucobol-trunk
\end{verbatim}\vspace{-2mm}
{\bf Build and Install:}\vspace{-3mm}
\begin{verbatim}
COBPREFIX=/opt/gnucobol
./build_aux/bootstrap
mkdir _build
cd _build
../configure --enable-cobc-internal-checks --enable-debug \
  --prefix $COBPREFIX --exec-prefix $COBPREFIX
make
sudo make install
\end{verbatim}\vspace{-2mm}
{\bf Configuration to use:}\vspace{-3mm}
\begin{verbatim}
COBPREFIX=/opt/gnucobol
PATH=$COBPREFIX/bin:$PATH
LD_LIBRARY_PATH=$COBPREFIX/lib:$LD_LIBRARY_PATH
export PATH LD_LIBRARY_PATH
\end{verbatim}
{\bf Supported Dialects:}\\
\begin{tabular}{ll}
\verb+acu+, \verb+acu-strict+ & ACUCOBOL-GT \\
\verb+bs2000+, \verb+bs2000-strict+ & BS2000 COBOL (Siemens) \\
\verb+cobol2002+ & Cobol ISO-2002 \\
\verb+cobol2014+ & Cobol ISO-2014 \\
\verb+cobol85+ & Cobol ANSI-1985 \\
\verb+default+ & Default config\\
\verb+ibm+, \verb+ibm-strict+ & IBM COBOL \\
\verb+mf+, \verb+mf-strict+ & Micro Focus COBOL  \\
\verb+mvs+, \verb+mvs-strict+ & MVS/VM COBOL \\
\verb+realia+, \verb+realia-strict+ & CA Realia II COBOL \\
\verb+rm+, \verb+rm-strict+ & RM-COBOL \\
\verb+xopen+ & OpenGroup Cobol 1991 C192\\
\end{tabular}

{\bf Environment Variables:}\\
~\hspace{-0.7em}\begin{tabular}{ll}
\verb+COB_PRE_LOAD=m1:m2+ & Lookup CALLs in \verb+m1,m2+\\ 
\verb+COB_CURRENT_DATE=YYYYDDMMHH24MISS+ & Date returned by ACCEPT\\ 
\verb+COB_LOAD_CASE=UPPER/LOWER+ & Lookup filenames\\ 
\verb+COB_LIBRARY_PATH=[...]+ & Lookup dynamic modules\\ 
\verb+COB_FILE_PATH=[...]+ & Lookup data files\\
\verb+COB_DISABLE_WARNINGS=true/false+ & Turn off warnings\\
%\verb+COB_ENV_MANGLE=true/false+ & ...\\
\verb+COB_SET_DEBUG=true/false+ & Turn on debug lines\\
\verb+COB_SET_TRACE=true/false+ & Turn on trace\\
%\verb+COB_TRACE_FILE=filename+ & ...\\
\verb+COB_STACKTRACE=true/false+ & Print stacktrace on abort\\
\verb+COB_DUMP_FILE=filename+ & Set dump file\\
\end{tabular}
  }

  \vbox{
\subsection{Compiler Options}

{\bf General Options:}\\
\begin{tabular}{ll}
\verb+cobc --help+ & Display help \\
\verb+cobc --version+ & Display version \\
\end{tabular}

{\bf Debugging compilation:}\\
\begin{tabular}{ll}
\verb+cobc -v [...]+ & Increase verbosity \\
\verb+cobc -q [...]+ & Suppress verbosity \\
\verb+cobc -### [...]+ & Do not execute sub-commands \\
\verb+cobc --save-temps [...]+ & Keep intermediary files \\
\verb+cobc -E [...]+ & Stop after pre-processing \\
\verb+cobc -fsyntax-only [...]+ & Stop after parsing \\
\verb+cobc -C [...]+ & Stop after C generation \\
\end{tabular}

{\bf Choosing Syntax:}\\
\begin{tabular}{ll}
\verb+cobc --free [...]+ & Use Free Format \\
\verb+cobc --fixed [...]+ & Use Fixed Format \\
\end{tabular}

{\bf Filenames conversions:}\\
\begin{tabular}{ll}
\verb+cobc -ffold-copy=[UPPER|LOWER] [...]+ & During COPY \\
\verb+cobc -ffold-call=[UPPER|LOWER] [...]+ & During CALL \\
\verb+cobc --ext CBL [...]+ & Extension for COPY\\
\end{tabular}
{\bf Language configuration:}\\
\begin{tabular}{ll}
\verb+cobc --std=DIALECT [...]+ & Use the specified dialect \\
\verb+cobc --conf=FILE [...]+ & Use the specified configuration \\
\end{tabular}
{\bf Output Configuration:}\\
\begin{tabular}{ll}
\verb+cobc -x -o FILE [...]+ & Generate executable FILE \\
\verb+cobc -c [...]+ & Gen. static module (\verb+.o+) \\
\verb+cobc -m [...]+ & Gen. dynamic module (\verb+.so+) \\
\verb+cobc -b [...]+ & Gen. dyn. library from static modules \\
\end{tabular}
{\bf Compiler Configuration:}\\
\begin{tabular}{ll}
\verb+cobc -I DIR [...]+ & Add DIR to include path\\
\verb+cobc -L DIR [...]+ & Add DIR to linking path\\
\verb+cobc -lXXX [...]+ & Link with libXXX.so\\
\verb+cobc -l:XXX.so [...]+ & Link with XXX.so\\
\verb+cobc -A OPT [...]+ & Pass option OPT to C compiler\\
\verb+cobc -Q OPT [...]+ & Pass option OPT to linker\\
\end{tabular}
{\bf Optimizations:}\\
\begin{tabular}{ll}
\verb+cobc -O2 [...]+ & More optimizations\\
\verb+cobc -O0 [...]+ & Disable all optimizations\\
\verb+cobc -fstatic-call [...]+ & Disable dynamic lookup of CALL\\
\end{tabular}
{\bf Warnings and Errors:}\\
\begin{tabular}{ll}
\verb+cobc -Wall [...]+ & Display all warnings\\
\verb+cobc -Wextra [...]+ & Display more warnings than \verb+-Wall+\\
\verb+cobc -fmax-errors=N [...]+ & Display N errors max\\
\verb+cobc -Werror [...]+ & Handle warnings as errors\\
\end{tabular}
{\bf Display Language Help:}\\
\begin{tabular}{ll}
\verb+cobc --list-reserved+ & List reserved keywords\\
\verb+cobc --list-intrinsics+ & List intrinsic functions\\
\verb+cobc --list-mnemonics+ & List mnemonics\\
\verb+cobc --list-system+ & List system routines\\
\verb+cobc --list-registers+ & List available registers\\
\end{tabular}
{\bf Debugging:}\\
\begin{tabular}{ll}
\verb+cobc -g [...]+ & Generate debugging information for \verb+gdb+\\
\verb+cobc -d [...]+ & Activate all error checks at execution\\
\verb+cobc -ftrace [...]+ & Generate limited execution trace\\
\verb+cobc -ftraceall [...]+ & Generate full execution trace\\
\end{tabular}
{\bf Execution:}\\
\begin{tabular}{ll}
\verb+cobcrun mod+ & Execute module \verb+mod.so+\\
\verb+cobcrun -M f mod+ & Execute entry \verb+f+ of module \verb+mod.so+\\
\end{tabular}
}

    \subsection{Debugging with {\tt gdb} and {\tt cbl-gdb}}

{\bf Installation:}\vspace{-3mm}
\begin{verbatim}
git clone https://gitlab.cobolworx.com/COBOLworx/cbl-gdb.git
cd cbl-gdb
git checkout master
make
sudo make install
\end{verbatim}\vspace{-2mm}
{\bf Create \verb+$HOME/.gdbinit+ file with:}\vspace{-3mm}
\begin{verbatim}
# enable use of COBOL cbl-dbg extension
add-auto-load-safe-path /usr/local/bin
set directories /usr/local/bin
set auto-load python-scripts on
\end{verbatim}\vspace{-2mm}

{\bf Build with \verb+cobcd+:}\vspace{-3mm}
\begin{verbatim}
export COBCD_COBC=/path/to/cobc
cobcd -x prog.cbl -o prog.exe
gdb prog.exe
\end{verbatim}\vspace{-2mm}
{\bf COBOL-specific commands:}
\begin{tabular}{ll}
\verb+cstart [ARGS]+ & Start the program with COBOL args \\
\verb+cbreak FILE:LINE+ & Add a COBOL breakpoint \\
\verb+ctbreak FILE:LINE+ & Add a one-time breakpoint \\
\verb+cwatch VARIABLE+ & Breakpoint on VARIABLE value \\
\verb+cnext N+ & Advance without entering PERFORM \\
\verb+auto-step+ & Step automatically \\
\verb+until-cobol+ & Execute until next COBOL module\\
\verb+finish-module+ & Execute until end of current module\\
\multicolumn{2}{l}{\tt finish-out-of-line-perform} \\
   & Execute until end of PERFORM \\
\verb+cprint VARIABLE+ & Print content of COBOL variable \\
\verb+cbacktrace+ & Print COBOL backtrace \\
\verb+local-backtrace+ & Print only local backtrace \\
\verb+cup+ & Select calling function\\
\verb+cdown+ & Select called function \\
\multicolumn{2}{l}{\tt add-symbol-file-cobol FILE} \\
  & Add a FILE containing symbols \\
\end{tabular}

  \subsection{VSCode Extension}

{\bf Install from VSIX:}
\url{https://cobolworx.com/pages/vsix.html}

{\bf Example of \verb+tasks.json+:}\vspace{-3mm}
\begin{verbatim}
{"version": "2.0.0",
 "tasks": [ { "label": "make",   "type": "shell",
   "command": "COBCD_COBC=/path/to/cobc cobcd -Q -Wl,-rpath=/path/to/lib -L/path/to/lib -x ${fileBasename}" }] }
\end{verbatim}\vspace{-2mm}
{\bf Example \verb+launch.json+:}\vspace{-3mm}
\begin{verbatim}
{ "version": "0.2.0",
  "configurations": [{
    "name": "cobc build and debug", "type": "cbl-gdb",
    "request": "launch",            "preLaunchTask": "make",
    "program": "${workspaceFolder}/${fileBasenameNoExtension}",
    "cwd": "${workspaceFolder}",    "arguments": ""
   }, {
    "name": "Attach cbl-gdb to Cobol process",
    "type": "cbl-gdb",           "cwd": "${workspaceFolder}",
    "solibs": "${env:PRIM_LIBRARY_PATH}", "request": "attach",
    "process_id": "${command:getAttachPID}" }]}
\end{verbatim}

\end{multicols}


\newpage

\makeheader{GnuCOBOL Compiler}
\begin{tabular}{p{15cm}p{10cm}}
  
  \noindent\begin{tabular}{|l|p{3cm}|p{7cm}|}\hline
    Type & Info & Size and Range \\ \hline
    \begin{tabular}{l}
      BINARY\\
      COMP\\
      COMP-4\\
    \end{tabular}
    &
    \begin{tabular}{l}
    Fixed Binary\\
    {\tt max\_int+1 = 0}\\
    {\tt min\_int-1 = 0}\\
    Range is PICTURE\\
    dependent\\
    \end{tabular}&
    \begin{tabular}{l}
 if {\sf binary-size} = {\tt 1-2-4-8}:\\
 9(1..2) 1 byte\\
 9(3..4) 2 bytes\\
 9(5..9) 4 bytes\\
 9(10..18) 8 bytes\\

 S9(1..2) 1 byte\\
 S9(3..4) 2 bytes\\
 S9(5..9) 4 bytes\\
 S9(10..18) 8 bytes\\
    \end{tabular}  \\ \hline
    \begin{tabular}{l}
    BINARY-C-LONG
  \end{tabular}
  &
    \begin{tabular}{l}
      System C {\tt long}\\
      No PICTURE clause
    \end{tabular}
    & 8 bytes on 64-bit arch, native signed range \\ \hline
    \begin{tabular}{l}
    BINARY-CHAR
\end{tabular}
&
    \begin{tabular}{l}
      System C {\tt char}\\
      No PICTURE clause
    \end{tabular}
    & 1 byte, -128..127   \\ \hline
    \begin{tabular}{l}
      BINARY-DOUBLE\\
      BINARY-LONG-LONG
\end{tabular}
    &
        \begin{tabular}{l}
          64-bit double float\\
          machine representation\\
          No PICTURE clause
        \end{tabular} &
        \\ \hline
    \begin{tabular}{l}
      BINARY-INT\\
      BINARY-LONG
    \end{tabular} &
        \begin{tabular}{l}
          32-bit native integer\\
          No PICTURE clause
          \end{tabular}
        &
        \begin{tabular}{l}
          4 bytes\\
        -2147483648[$-2^{31}$]..+2147483647[$2^{31}-1$]\\
        \end{tabular}
    \\ \hline
    \begin{tabular}{l}
      DISPLAY\\
    \end{tabular}
    & signed integers until 38 digits &
        \begin{tabular}{l}
    1 byte per digit in PICTURE\\
    sign included\\
    \end{tabular} 
    \\ \hline
    \begin{tabular}{l}
      PACKED-DECIMAL\\
      COMP-3
    \end{tabular}
    & signed integers until 38 digits &
        \begin{tabular}{l}
          4 bits for the sign, 4 bits per digit in PICTURE\\
          1 digit: 1 byte\\
          2-3 digits: 2 bytes\\
          4-5 digits: 3 bytes\\
          ...\\
          36-37 digits: 19 bytes\\
          38 digit: 20 bytes\\
    \end{tabular} 
    \\ \hline
    \begin{tabular}{l}
      COMP-5
    \end{tabular}&
    \begin{tabular}{l}
      Binary in native\\
      byte order\\
      Range is byte\\
      dependent\\
      {\tt max\_int+1 = min\_int}
      \end{tabular} & 
    \begin{tabular}{l}
 if {\sf binary-size} = {\tt 1-2-4-8}:\\
 9(1..2) 1 byte 0..255[$2^8-1$]\\
 9(3..4) 2 bytes 0..65535[$2^{16}-1$]\\
 9(5..9) 4 bytes 0..4294967295[$2^{32}-1$]\\
 9(10..18) 8 bytes 0..[$2^{64}-1$]\\

 S9(1..2) 1 byte -128[$-2^7$]..+127[$2^7-1$]\\
 S9(3..4) 2 bytes -32768[$-2^{15}$]..+32767[$2^{15}-1$]\\
 S9(5..9) 4 bytes \\
 -2147483648[$-2^{31}$]..+2147483647[$2^{31}-1$]\\
 S9(10..18) 8 bytes [$-2^{63}$]..[$2^{63}-1$]
  \end{tabular}
  \\ \hline
  \begin{tabular}{l}
    UNSIGNED-PACKED\\
    COMP-6
  \end{tabular}
  & unsigned integers until 38 digits
  &
        \begin{tabular}{l}
          4 bits per digit in PICTURE\\
          1-2 digits: 1 byte\\
          3-4 digits: 2 bytes\\
          ...\\
          37-38 digits: 19 bytes\\
    \end{tabular} 
  \\ \hline
    \begin{tabular}{l}
      COMP-X
  \end{tabular}
    &
    \begin{tabular}{l}
      Interprets PIC X(n) \\
      as an unsigned integer\\
      until 38 digits
    \end{tabular}
    &
    1 byte per digit
    \\ \hline
    \begin{tabular}{l}
      INDEX
    \end{tabular}
    &
    \begin{tabular}{l}
      signed integer\\
      no PICTURE clause
    \end{tabular}&
    \begin{tabular}{l}
      4 bytes \\
      -2147483648[$-2^{31}$]..+2147483647[$2^{31}-1$]
      \end{tabular}
    \\ \hline
    \begin{tabular}{l}
      FLOAT-SHORT\\
      COMP-1
    \end{tabular}
    & Single precision float
    & 4 bytes
    \\ \hline
    \begin{tabular}{l}
      FLOAT-LONG\\
      COMP-2
    \end{tabular}
    & Double precision float
    & 8 bytes
    \\ \hline
  \end{tabular}

&

The default for USAGE BINARY and COMP is big-endian (see {\tt
  binary-byteorder}). Little-endian on Intel for COMP-5, BINARY-CHAR,
BINARY-SHORT, BINARY-LONG, BINARY-DOUBLE.

BIT is not implemented.
\end{tabular}

\end{document}

%%% Local Variables:
%%% mode: latex
%%% TeX-master: t
%%% End:

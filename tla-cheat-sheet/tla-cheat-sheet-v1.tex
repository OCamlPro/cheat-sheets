\documentclass[10pt,landscape]{article}
\usepackage{../common/cheat-style}
\usepackage{url}
\usepackage{hyperref}

\begin{document}

\setlength{\premulticols}{1pt}
\setlength{\postmulticols}{1pt}
\setlength{\multicolsep}{1pt}
\setlength{\columnsep}{2em}
\def\csRevision{1}
\def\csSubtitle{TLA}
\makeheader{The TLA cheat-sheet}

\newcommand{\expr}{\textit{expr}}
\newcommand{\IN}{\textbf{$~\backslash$in}}
\newcommand{\NOTIN}{\textbf{$~\backslash$notin}}
\newcommand{\SUBSETEQ}{\textbf{$~\backslash$subseteq}}
\newcommand{\SQSUBSETEQ}{\textbf{$~\backslash$sqsubseteq}}
\newcommand{\UNION}{\textbf{$~\backslash$union}}
\newcommand{\INTER}{\textbf{$~\backslash$intersect}}
\newcommand{\DIV}{\textbf{$~\backslash$div}}
\newcommand{\LEQ}{\textbf{$~\backslash$leq}}
\newcommand{\GEQ}{\textbf{$~\backslash$geq}}
\newcommand{\E}{$\backslash E$}
\newcommand{\A}{$\backslash A$}

\begin{multicols}{3}
  \section{TLA Standard}
  \vbox{
    \subsection{Module format}
    \begin{tabular}{p{0.15\textwidth}|p{0.14\textwidth}}
      -{}-{}-{}- MODULE filename -{}-{}-{}- & Header of any module\\
      \textbf{EXTENDS} mname & Adds the content of module to the current one\\
      \textbf{CONSTANTS} c1, ... & Defines constant names for the module\\
      \textbf{VARIABLES} v1, ... & Defines global variables for the module\\
      M == \textbf{INSTANCE} mname & Creates a namespace for an imported module\\

      %   INSTANCE ``m'' WITH var1 $\leftarrow$ var ... &|& Links variables/constants
      %   of the current module and the imported one\\

      vname $==$ \expr & defines a global value\\

      fun(arg1, ...) $==$ \expr & Defines a global function\\

      \textbf{RECURSIVE} fun(\_,...) & Declares the future definition of a recursive function\\

      fun[x \IN~$S$] == \expr & Defines a function whose arguments belong
      to a given set (may be recursive without declaring beforehand)\\

      \textbf{LOCAL} name... & Defines a local value or function\\

      \textbf{ASSUME} $P$ & Asserts $P$ as an assumption\\

      $\backslash *$ & Starts a single-line comment\\

      ==== & Footer of any module\\
    \end{tabular}

    \subsection{Generic expressions}

    \begin{tabular}{p{0.15\textwidth}|p{0.14\textwidth}}
      $TRUE$, $FALSE$ & Booleans\\
      $\backslash$/, /$\backslash$, $\sim$ & Boolean operators \textit{or}, \textit{and}, \textit{not} \\
      $<<val1, ...>>$ & Sequences/tuples\\
      \null[field |-> \expr, ...] & Records \\
      \null[r \textbf{EXCEPT} !.field1 = \expr, ...] & Record update \\
      r.field & Record access\\
      \null[x\IN~$S$ |-> \expr] & Functions \\
      \null[f \textbf{EXCEPT} ![x] = \expr, ...] & Function update \\
      f[x] & Function call\\
      \{x1, x2, ...\} & Sets\\
      \{\expr : x\IN~$S$\}& Mapping of set\\
      \{x\IN~$S$ : p \}& Filtering of set\\
      \textbf{DOMAIN} $f$ & Domain of the function/tuple/sequence $f$\\
      \textbf{LET} v == \expr~\textbf{IN} ... & Local variable\\
      \textbf{IF} \expr\\
      \textbf{THEN} \expr\\
      \textbf{ELSE} \expr & Conditional statement \\
    \end{tabular}
  }

  \vbox{
    \begin{tabular}{p{0.15\textwidth}|p{0.14\textwidth}}
      Module\textbf{!}value & Use the value defined in a namespace\\
      \textbf{CASE} $p_1$ -> $\expr_1$\\
      \textbf{[]} $p_2$ -> $\expr_2$ ...\\
      \textbf{OTHER} -> $\expr$ &
      Selects an $\expr_i$ such that $p_i$ is $TRUE$, otherwise selects \expr\\
    \end{tabular}

  \subsection {Boolean Predicates}
    \begin{tabular}{p{0.15\textwidth}|p{0.14\textwidth}}
      \E~$x$\IN~$S$: $p$ & Existential quantifier\\
      \A~$x$\IN~$S$: $p$ & Universal quantifier\\
      \textbf{CHOOSE} x\IN~$S$: p & Selection in set\\
      $p => q$ & Implication\\
      $p <=> q$ & Equivalence 
    \end{tabular}

  \subsection {Action Predicates}
    \begin{tabular}{p{0.15\textwidth}|p{0.14\textwidth}}
      $e'$ & The value of $e$ after a step\\
      $[A]_e$ & $A$ or $e' = e$\\
      $<A>_e$ & $A$ and $e' \neq e$\\
      $\textbf{UNCHANGED}~e$  & $e$ did not change\\
      $\textbf{ENABLED}~A$ & $A$ is possible
    \end{tabular}

  \subsection {Temporal Predicates}
    \begin{tabular}{p{0.15\textwidth}|p{0.14\textwidth}}
      []~$p$ & Always $p$\\
      <>~$p$ & Eventually $p$\\
      $p \sim> q$ & $p$ leads to $q$\\
      $\textbf{WF}_e(A)$ & Weak Fairness for action $A$\\
      $\textbf{SF}_e(A)$ & Strong Fairness for action $A$\\
    \end{tabular}
    
    \section{Useful modules}

    \subsection{Sequences}
 
    \textbf{Sequences are $1$-indexed}
    
    \begin{tabular}{p{0.15\textwidth}|p{0.14\textwidth}}
      $s[i]$ & $i^{th}$ element of the sequence\\ 
      $\textbf{Head}(s)$ & First element of a sequence\\
      $\textbf{Tail}(s)$ & The sequence without its head\\
      $\textbf{Append}(s, i)$ & Adds $i$ at the end of sequence $s$\\
      $s_1~\backslash o~s_2$ & Concatenation \\
      $\textbf{Len}(s)$ & Length of a sequence \\
      $\textbf{Seq}(S)$ & Sequences of elements of set $S$\\
    \end{tabular}

    \subsection{FiniteSets}
    
    \begin{tabular}{p{0.15\textwidth}|p{0.14\textwidth}}
      $x \IN~S$ & $x$ is in set $S$\\
      $x \NOTIN~S$ & $x$ is not in set $S$\\
      $S \SUBSETEQ~T$ & $S$ is a subset of $T$\\
      $S \UNION~T$ & Union of sets\\
      $S \INTER~T$ & Intersection of sets\\
      $S~\backslash~T$ & $S$ without elements of $T$\\
      $\textbf{SUBSET}~S$ & All the subsets of $S$\\
      $\textbf{UNION}~S$ & Flatten sets of sets\\
      $\textbf{IsFiniteSet}(S)$ & $TRUE$ if $S$ if finite\\
      $\textbf{Cardinality}(S)$ & Number of elements of $S$
    \end{tabular}

  }

  \vbox{
    \subsection{Naturals, Integers}
    
    \begin{tabular}{p{0.15\textwidth}|p{0.14\textwidth}}
      $\textbf{Nat}$, $\textbf{Int}^*$ & Sets of numbers\\
      $+$, $-^*$, $*$, $\DIV$ & Arithmetical operators\\
      $x ^ y$ & $x$ to the $y$\\
      $<$, $>$, $\LEQ$, $\GEQ$ & Comparison operators\\
      $\%$ & Modulo\\
      $ x .. y $ & $\{x, x+1, ..., y\}$
    \end{tabular}
    $\null^*$ only available in the Integer module
    
    \subsection{Reals}
    
    \begin{tabular}{p{0.15\textwidth}|p{0.14\textwidth}}
      $\textbf{Real}$ & Set of reals\\
      $/$ & Real division\\
      $\textbf{Infinity}$ & Value greater than any real (NOT A REAL)
    \end{tabular}

    \subsection{TLC}
    \begin{tabular}{p{0.15\textwidth}|p{0.14\textwidth}}
      $\textbf{Print}(msg,~val)$ & Prints $msg$, then returns $val$\\
      $\textbf{Print}(msg)$ & $\textbf{Print}(msg, TRUE)$\\
      $\textbf{Assert}(val, out)$ & Prints $out$ and fails iff $val$ is $FALSE$\\
      $d~\textbf{:>}~e$ & $f(d) = e$\\
      $f~\textbf{@@}~g$ & Union of functions\\
      $\textbf{SortSeq}(s, Op(\_,\_))$ & Sorts a sequence\\
      $\textbf{ToString}(v)$ & String representation of $v$\\
      $\textbf{TLCEval}(v)$ & Forces evaluation of $v$ 
    \end{tabular}
    \subsection{Bags}
    A bag is a set that can contain multiple (finite) copies of the same
    element.

    ~
    
    \begin{tabular}{p{0.15\textwidth}|p{0.14\textwidth}}
      $\textbf{EmptyBag}$ & The empty bag \\
      $\textbf{IsABag}(B)$ & Checks if $B$ is a bag\\
      $\textbf{BagToSet}(B)$ & The set of bag elements\\
      $\textbf{SetToBag}(S)$ & The bag of set elements\\
      $\textbf{BagIn}(B,e)$ & Checks if $e$ is in the bag\\
      $(+)$, $(-)$ & Union, disjunction\\
      $\textbf{BagUnion}(S_B)$ & Union of set of bags\\
      $B1 \SQSUBSETEQ~B2$ & Subset\\
      $\textbf{SubBag}(B)$ & Set of all sub-bags\\
      $\textbf{BagOfAll}(F(\_), B)$ & Mapping on bags\\
      $\textbf{BagCardinality}(B)$ & Size of a bag\\
      $\textbf{CopiesIn}(e, B)$ & Number of $e$ in the bag $B$
    \end{tabular}
    \subsection{Json}
    \begin{tabular}{p{0.15\textwidth}|p{0.14\textwidth}}
      $\textbf{ToJson}(v)$ & Returns $v$ as a Json string\\
      $\textbf{ToJsonArray}(v)$ & Same, but for a sequence\\
      $\textbf{ToJsonObject}(v)$ & Returns a Json object\\
      $\textbf{JsonSerialize}(file^*, v)$ & Writes $v$ as a (plain) Json in $file$\\
      $\textbf{ndJsonSerialize}(file^*, v)$ & Same, but Json is newline delimited\\
      $\textbf{JsonSerialize}(file^*)$ & Returns the content of $file$\\
      $\textbf{ndsonSerialize}(file^*)$ & Same, but values must be newline delimited\\
    \end{tabular}
    $\null^*$ file name must be absolute
  }
\end{multicols}

\clearpage

\makeheader{The TLA cheat-sheet}
\section{Creating a model}
\begin{multicols}{3}
  \vbox{
    \subsection{Counter.tla}
    \begin{verbatim}
Implements a simple counter

---- MODULE Counter ----

EXTENDS Naturals

\* An unknown constant
CONSTANTS MAX

\* The variables of our model
VARIABLES counter, reset

\* The initial state (must be finite)
Init == 
  /\ counter \in 0..MAX
  /\ reset \in {TRUE, FALSE}

\* If `reset' is set, then counter is reinitialized
Incr ==
  /\ ~reset
  /\ reset' \in {TRUE, FALSE}
  /\ counter' = counter + 1

\* If `reset' is set, then counter is reinitialized
Reset ==
  /\ reset
  /\ reset' \in {TRUE, FALSE}
  /\ counter' = 0


\* The Next state predicate
Next ==
  \/ Incr
  \/ Reset

\* The specification of our  model:
\*   - starts by Init
\*   - Next is the next state predicate, but variables
\      are allowed not to change between steps
Spec == 
  /\ Init
  /\ [][Next]_<<counter, reset>>
====
    \end{verbatim}
  }
\vbox{
  \subsection{Props.tla}
  \begin{verbatim}
Properties on the counter model
---- MODULE Props ----

CONSTANT MAX

VARIABLES counter, reset

LOCAL INSTANCE Counter WITH
  MAX <- MAX,
  counter <- counter,
  reset <- reset


\* Invariants

\* Variable `counter' is always positive
AlwaysPositive == counter >= 0

\* Temporal Properties

\* If `reset' happens, then `counter' will be 0
ResetLeadsToZero ==
    reset ~> counter = 0

\* Either:
\*   in the future, `reset' will never be triggered;
\*   or `counter' repeatedely reaches 0.
CounterRuns ==
  \/ <>[](~reset)
  \/ []<>(counter = 0)
====
    \end{verbatim}
  \subsection{Props.cfg}
  \begin{verbatim}
CONSTANT
  MAX = 0

SPECIFICATION
  Spec

INVARIANTS
  AlwaysPositive

PROPERTIES
  ResetLeadsToZero
  CounterRuns
  \end{verbatim}
  }

  \vbox{
    \subsection{Output}

    \begin{verbatim}
Temporal properties were violated.
The following behavior constitutes a counter-example:
1: <Initial predicate>
/\ counter = 0
/\ reset = FALSE
2: <Action line 7, col 1 to line 10, col 16 of module Props>
/\ counter = 1
/\ reset = FALSE
3: <Action line 7, col 1 to line 10, col 16 of module Props>
/\ counter = 2
/\ reset = FALSE
4: <Action line 7, col 1 to line 10, col 16 of module Props>
/\ counter = 3
/\ reset = TRUE
5: Stuttering
    \end{verbatim}
}


\end{multicols}
\end{document}

%%% Local Variables:
%%% mode: latex
%%% TeX-master: t
%%% End:

  
  \vbox{
    \subsection{More...}
  }
  
  \vbox{
    \subsection{Even more..}

  }
